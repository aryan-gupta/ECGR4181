\documentclass{article}
\usepackage{listings}
\usepackage{pdfpages}
\graphicspath{ {.} }

\usepackage{fancyhdr}
\usepackage[pdftex,
	pdfauthor={Aryan Gupta},
	pdftitle={ECGR4181 HW1 Report},
	pdfsubject={Homework Report},
	pdfkeywords={},
	pdfproducer={Latex with hyperref},
	pdfcreator={}
]{hyperref}
\usepackage[margin=1in]{geometry}
\hypersetup{colorlinks=true, linkcolor=black,urlcolor=black}
\usepackage[margin=1in]{geometry}
\usepackage{lastpage}
\usepackage[margin=1in]{geometry}
\usepackage{fancyhdr}
\usepackage{amsmath,amsfonts, amsthm}
\usepackage{float}
\usepackage{graphicx}

\lstset{
	basicstyle=\ttfamily,
	columns=fullflexible,
	frame=single,
	breaklines=true,
	postbreak=\mbox{$\hookrightarrow$\space},
}

\begin{document}
	\section{Pintools}
		The first part of this project was to use Pin tools to analyze the TLB hierarchy of our processor. My processor only had one set of TLBs, one for instructions and another for data. This made it very hard to analyze how effective the TLB actually was. The TLB effectively reduces the number of memory access when using virtual addresses. The TLB stores virtual address to physical address translation in a small buffer and is used as a cache of sorts for this lookup. In theory, without a TLB every memory access could translate to multiple memory accesses for the OS in order to translate the address. Here in Table \ref{tlb:table}, you can see that Dhrystone had to access the memory over 250 million times when accessing instructions. If there was no TLB, there would be upwards of 500 million memory access, and considering the latency for a modern processor to access RAM, this latency would build up and slow down the processor while its waiting for memory and not doing anything. In the results, you can see that there was only a small percent of a percent TLB miss, this means that the TLB algorithm is very efficient and removed the latency effect of virtual addresses. A similar result is obtained for the data TLB. The test was run 3 times and an average was taken. 
		
		\begin{table}[H]
			\label{tlb:table}
			\centering 
			\begin{tabular}{lllll}
				    & Test1 & Test2 & Test3 & Average \\ 
				Dps & 38461.5 & 40000 & 38461.5 & 38974.33333 \\ 
				& & & & \\
				ITLB Hits & 263245913 & 263245896 & 263245731 & 263245846.7 \\
				ITLB Miss & 135 & 135 & 135 & 135 \\
				ITLB Total & 263246048 & 263246031 & 263245866 & 263245981.7 \\
				& & & & \\
				DTLB Hits & 124077146 & 124077142 & 124077084 & 124077124 \\
				DTLB Miss & 269 & 255 & 269 & 264.3333333 \\
				DTLB Total & 124077415 & 124077397 & 124077353 & 124077388.3
			\end{tabular}
		\end{table}
\end{document}